\documentclass{article}

\usepackage[utf8]{inputenc}
\usepackage[english]{babel}
\usepackage{dsfont}
\usepackage{amsmath}
\usepackage{amssymb}
\usepackage{graphicx}% Include figure files
\usepackage{dcolumn}% Align table columns on decimal point
\usepackage{bm}% bold math
\usepackage{amsmath}
\usepackage{varioref}
\usepackage{booktabs}
\usepackage[bottom]{footmisc}

\usepackage{wasysym}

\usepackage{physics}

\usepackage{algpseudocode}
\usepackage{listings}

\usepackage{booktabs}

\usepackage{tikz}

\newcommand{\RN}[1]{%
  \textup{\uppercase\expandafter{\romannumeral#1}}%
}



\usepackage[utf8]{inputenc}

\usepackage{natbib}
\usepackage{graphicx}
\usepackage[]{hyperref}
\usepackage[]{physics}
\usepackage[]{listings}
\usepackage[T1]{fontenc}
\usepackage{color}
\usepackage{float}
\usepackage{soul}
\lstset{
  backgroundcolor=\color{white}, % requires \usepackage{color} or \usepackage{xcolor}
  basicstyle=\footnotesize,
  breakatwhitespace=false,
  breaklines=true,
  captionpos=b,
  commentstyle=\color{green},
  deletekeywords={...},
  escapeinside={\%*}{*)},
  extendedchars=true,
  frame=single,
  keepspaces=true,
  keywordstyle=\color{blue},
  language=c++,
  otherkeywords={*,...},
  rulecolor=\color{black},
  showspaces=false,
  showstringspaces=false,
  showtabs=false,
  stepnumber=2,
  %stringstyle=\color{pink},
  tabsize=4,
}

\title{The Ising model and the Metropolis algorithm}
\author{
  Brandt, Samuel\\
 % \texttt{}
  \and
  Davidov, Aleksandar\\
  \textcolor{blue}{\href{https://github.com/aleksda/FYS4150/}{\texttt{github.com/aleksda}}}
  \and
  Hemaz, Said\\
  % \texttt{first2.last2@xxxxx.com}
}
\date{November 2019}

\begin{document}

\maketitle

\section{Abstract}

The goal of this project is to compute the Curie temperature  $T_c$ as the system undergoes phase transition, with the help of the Ising Model and the Metropolis algorithm. Through the calculation of a specific lattice with the following parameters: energy $E$, absolute magnetization  $|M|$, heat capacity   $C_V$ and suceptiblility  $\chi$, the analytical and numeric values where in good correspondence, given the right amount of Monte-Carlo cycles.
\newpage
\tableofcontents
\newpage
\section{Introduction}

In this project we will study a model that simulates phase transitions. The particular simulation we will study is known as the Ising model. This is a mathematical model in statistical mechanics, used to simulate the magnetic properties of a material at a given temperature. The Ising model we are going to use is a simplified version where no external field will interact with the lattice. \\\

We will be using the Ising model to simulate a two-dimensional ferromagnetic material. At a critical temperature, the Ising model will show a phase transition from a magnetic state to a non-magnetic state. This is then a binary system that holds only two values, in our case it will be spin up or spin down. Starting from this two dimensional lattice we will begin with describing the theory of the Ising model, and the workings of our algorithm of choice, the Metropolis algorithm. Then we find the analytical expressions of partition function and the expectations values for  $E$ (energy), $|M|$ (magnetic moment),  $C_V$(specific heat capacity) and  $\chi$ (magnetic susceptibility) with respect to the temperature $T$, using periodic boundary conditions. These results will be used in the numerical computations to uncover the critical temperature, with the help of parallelization. At last the results will be discussed.

\section{Theory}

\subsection{The Ising model }\label{const_mot}

The Ising model consists of discrete variables that describe magnetic dipole moments in terms of spin up or spin down. In our simplified version, the Hamiltonian of a  two dimensional  model is given by the equation:
\begin{equation}\label{eq:Hamiltonian}
H = -J\sum_{\langle ij \rangle}S^z_i S^z_j,
\end{equation}
where $\langle ij \rangle$ denotes a sum over the nearest neighbours of each spin. $J$ is the coupling constant that describes the interaction between the neighbouring spins. We will assume $J$ to be positive, in other words that we have a state of ferromagnetic ordering, which means the the neighbouring spins are aligned, and thus the energy will be lower. 
The two dimensional Ising model on an infinite square lattice is  analytically solved, by L. Onsager. The critical temperature was calculated to be
\begin{equation}
k_B T_C / J = \frac{2}{\ln (1 + \sqrt{2})} \approx 2.2692.
\end{equation}

\subsection{Analytical expressions}\label{const_mot}

First we start of by calculating the partition function, since it  is needed to get analytical expressions for physical values like the susceptibility and heat capacity.
For a $2 \times 2$ lattice with periodic boundary conditions, the possible energies and magnetizations are listed with the degeneracies in Table 
\ref{table:2times2}. 

% Please add the following required packages to your document preamble:
% \usepackage{booktabs}
\begin{table}[]
\caption{The number of spins pointing up on a $2\times 2$ lattice with corresponding degeneracies, energies and magnetizations. The energies are calculated with periodic boundary conditions.}
\label{table:2times2}
\centering
\begin{tabular}{@{}cccc@{}}
\toprule
$\quad$\ spins$\quad$ & deg. & $E [J]$   & $M$  \\ \midrule
$4$     & $\qquad 1\qquad$  & $\quad-8\quad$ & $4$  \\
$3$     & $4$  & $0$   & $2$  \\
$2$     & $4$  & $0$   & $0$  \\
$2$     & $2$  & $+8$ & $0$  \\
$1$     & $4$  & $0$   & $-2$ \\
$0$     & $1$  & $-8$ & $\qquad -4\qquad$ \\ 
\bottomrule
\end{tabular}
\end{table}

The energy is calculated from Eq.(1), and the magnetization is the sum of all the spins in the following equation,
\begin{equation}
M = \sum_{i = 1}^{N} S^z_i,
\end{equation}
where $N$ is the total number of spins.

With the help of the table above, we can calculate the mean energy $\ev{E}$, mean energy squared $\ev{E^2}$, (absolute) magnetization $\ev{\abs{M}}$ and  $\ev{M^2}$. The thermal average of a physical quantity is given by
\begin{equation}
\ev{A} = \frac{1}{Z} \sum_i A_i e^{-\beta E_i},
\end{equation}
where $i$ is the number of micro states, $\beta = 1/k_B T$ and $Z$ is the partition function, which in our case of $2 \times 2$ lattice is,
\begin{equation}
Z = \sum_i e^{-\beta E_i} = 4\left(3 + \cosh (8\beta J)\right).
\end{equation}

We then get
\begin{align}\label{eq:expvals}
\ev{E} &= \frac{1}{Z} \sum_i E_i e^{-\beta E_i} = -\frac{1}{Z}32J\sinh (8\beta J),\\
\ev{E^2} &= \frac{1}{Z} \sum_i E_i^2 e^{-\beta E_i} = \frac{1}{Z}256J^2\cosh (8\beta J),\\
\ev{\abs{M}} &= \frac{1}{Z} \sum_i \abs{M_i} e^{-\beta E_i} = \frac{1}{Z}8\left(2 + e^{8\beta J}\right),\\
\ev{M^2} &= \frac{1}{Z} \sum_i M_i^2 e^{-\beta E_i} = \frac{1}{Z}32\left( 1 + e^{8\beta J}\right).
\end{align}

From these expectation values, we can calculate the heat capacity at constant volume and the susceptibility of the system from the following relations  \cite{hjorten}:
\begin{equation}\label{eq:heatcap}
C_V(T) = \frac{1}{k_BT^2}\left(\ev{E(T)^2} - \ev{E(T)}^2\right),
\end{equation}
\begin{equation}\label{eq:suscept}
\chi(T) = \frac{1}{k_B T}\left(\ev{M(T)^2} - \ev{M(T)}^2\right).
\end{equation}

\subsection{Phase transitions }\label{const_mot}

Phase transitions takes place  in the critical point, which is marked by the critical temperature $T_C$. The material then goes from being a ordered ferromagnetic material to a disordered paramagnetic material. 

The behaviour around the critical temperature can be connected to the thermodynamical  potential of  Helmholtz' free energy

\begin{equation}
    F = \langle E \rangle - TS = - k_B TlnZ
\end{equation}

where Z is the partition function , T is the temperature, $k_B$ is the Boltzmann constant and S is the entropy. For simplification  $k_B $ will be set to one.  This equation shows the struggle between energy minimization and the maximizing of entropy. For our two dimensional system, the desired outcome would be a minimzation of the energy. In this case, the minimal energy state is the ordered ferromagnetic state and also the the state of minimal chaos, hence minimal entropy.  \newline

The specific heat $C_V$ is defined via the second derivative of F: $C_V = - \frac{1}{k_B T^2}\frac{\partial^2(\beta F)}{\partial \beta ^2}$, and $\beta = \frac{1}{k_B T}$. Since the heat capacity and also the magnetic susceptibility  $\chi$  diverges, the Ising model shows a second order phase transition.. Since both these magnitudes are temperature dependent , we also expect them to be affected when we reach $T_C$. \newline




\section{Algorithms}

\subsection{Monte-Carlo methods }\label{const_mot}
Monte Carlo (MC) simulations has their basis on Markov chains, where the configuration evolves over time by multiplications of a stochastic matrix, and converges towards a steady-state.

MC simulations utilizes probability distribution , which in our case is Boltzmann statistics. In order to find all the probabilities, the partition function has to be calculated , which takes a lot  of computer power , especially as the number of lattice points increase. This is where the Metropolis algorithm becomes useful, because the algorithm only takes in account the ratios between the probabilities , in this case the partition function cancels out and we only have to compute the Boltzmann factor.

\subsection{The Metropolis algorithm}\label{const_mot}

The Metropolis algorithm calculates the change in energy, $\Delta E$, using this formula:

\begin{equation}
\Delta E = 2JS^1_l\sum_{\langle k \rangle}S_k,
\end{equation}
where $S^1_l$ is the value of the spin \textit{before} it is flipped and $S_k$ are the neighbour spins. 






The Metropolis algorithm can be broken down to the following steps.
\begin{enumerate}
\item The system is initialised by an initial state of energy $E_0$ and magnetization $M_0$ which can be  generated randomly. 
\item The initial system is changed by flipping the spin of a random site. One then calculates the energy of this new state $E_{new}$.
\item The difference in energy between the two states $\Delta E =  E_{new} -E_0$ is calculated. Since we only take into account the closest neighbour interactions, there are only five  possible values for $\Delta E$ when one spin is flipped. These are listed in the table above. This means, we can spare a lot of FLOPS by simply precalculating the factor  $e^{-\beta\Delta E}$  for each temperature  
\item If $ \Delta E < 0$ the new configuration is accepted, meaning that the energy has  lowered , and the system is moving towards an energy minimum. Note that the current value  for energy and magnetization is updated. We therefore have to update the expectation values for $E$, $E^2$, $M$, $M^2$ and $\abs{M}$, before the next MC cycle repeats.
\item If $ \Delta E > 0$ , compute and compare  $w =   e^{-\beta\Delta E}$  to a random number $r$ $\in$ [0, 1]. If $r < w$ the state is accepted, else the initial state is kept.
\item The expectation values are updated. And the steps 2-5 will be repeated until a satisfactory good steady state will be reached.

\end{enumerate}
Other properties of the Metropolis algorithm is that it meets both ergodicity and detailed balance, which are requirements needed for convergence towards equilibrium.

As it takes some time for the configuration to converge towards equilibrium, the results should get better the more MC cycles that is used.



\section{Code and implementation}
In order to produce the numerical values relevant for this project, we used C++ for the computations while the data analysis was done with Python. All the source code used was is one file. Then, you call on the methods, from the main function. This is also where cycles and lattice size is specified. The source code was written using object-orientation as that is the type programming I am most familiar with. Besides that, object-orientation makes it easier to reuse values as these get as they get updated for each instance of the class. 
\\
The most important method in our class is of course the metropolis algorithm. This method is called in a method called simulate that also prints all of the data we need to a single file. Besides that, there is also a method that generates a lattice with assigned with random spin values. For more information, see the isingModel.cpp file. 
\\
As mentioned, Python was used for the data analysis and visualization. The reason for this is that Python has a wide range of libraries for analysis and visualization that makes it more suited for this kind of computing. The text-files generated from the simulate method are used in the Python files for analysis. 

\section{Results}

\begin{figure}[H]
	\centering
	\includegraphics[width = \textwidth]{4b).png}
	\centering
	\caption{Plot that shows the numerical values as a blue line  and the analytical values as a red line the expectation value of the energy, absolute value of the magnetization, as well as the heat capacity and susceptibility for a $2\times 2$ lattice as functions of temperature. This was calculated using $10^6$ MC cycles without the system reaching equilibrium. We can the numerical and analytical solutions are overlapping quite well with this number of cycles  }
	\label{fig: integration limits Gauss-Legendre}
\end{figure}



\begin{figure}[H]
	\centering
	\includegraphics[width = \textwidth]{ordered.png}
	\centering
	\caption{The expectation value of the absolute value of the energy and magnetization as a function of MC cycles for a $20 \times 20$  lattice. The simulation have been done for two different temperatures with and ordered initial condition . Ordered means that all spins starts pointing up ($S^z = 1$). }
	\label{fig: integration limits Gauss-Legendre}
\end{figure}

\begin{figure}[H]
	\centering
	\includegraphics[width = \textwidth]{unordered.png}
	\centering
	\caption{The expectation value of the absolute value of the energy and magnetization as a function of MC cycles for a $20 \times 20$ lattice. The simulation have been done for two different temperatures with  initial disordered configuration. Disordered means that all spins starts configuration is chosen randomly. }
	\label{fig: integration limits Gauss-Legendre}
\end{figure}


\begin{figure}[H]
	\centering
	\includegraphics[width = \textwidth]{acceptedStates.png}
	\centering
	\caption{Figure showing that the number of accepted states for a high-temperature system ($T=2.4$) has a proportional relation , while the accepted states for for a lower-temperature system ($T=1$) has a constant relation at equilibrium.}
	\label{fig: integration limits Gauss-Legendre}
\end{figure}

\begin{figure}[H]
	\centering
	\includegraphics[width = \textwidth]{probDensity.png}
	\centering
	\caption{Probability densities for temperature $T=1$ and temperature $T=2.4$}
	\label{fig: integration limits Gauss-Legendre}
\end{figure}
\begin{figure}[H]
	\centering
	\includegraphics[width = \textwidth]{4e.png}
	\centering
	\caption{ Plot demonstrating the comparison of the runtime with no flag and with O2 flag  }
	\label{fig: integration limits Gauss-Legendre}
\end{figure}



\section{Discussion}
From figure 2 and 3 we can see that the speed of reaching equilibrium depends on the initial value of low temperatures , but is less significant when it comes to higher temperatures. This is because for lower temperatures , the ordered state is preferable. Order in this case means most the spins are pointing in one directions, also we have a ferromagnetic material. Starting from a lower temperature then means we have a an existing ordered system that is in other words in a equilibrium. Therefore for low temperatures equilibrium is reached instantaneously. Because of the accepting rule of the Metropolis algorithm, we stay in equilibrium. This is because the rule gives a small probability for flipping a spin when in steady-state, even if it did, it is highly likely it will be flipped back.  From a disordered standpoint , or a higher temperature we are far away from equilibrium, since all configurations of spin will be accepted. This can be seen in the blue graphs of  $T = 1.0J/k_B$. The configuration converges towards steady state, but will suddenly flip to a much more likely configuration and towards the more realistic equilibrium. The amount of MC cycles we required for equilibrium seems to be around $10^8$ to above a million cycles. Because of lack of computing power we did not have plots that shows above this amount of cycles.

As for higher temperature of $T = 2.4J/k_B$, the equilibrium speed is independent on the initial ordering. Here there should be a mostly non magnetized configuration , half of the spins point up and the other half pointing down. The magnetization is not zero as the figure shows for steady-state, the same applies for the small lattice. Since the the critical temperature for this lattice is still above  2.4J/kB, this wont be a completely disordered state. This is why the disordered and the ordered is not on steady-state . At this point , they will both reach steady state at the same rate.

 \\\



Figure 4 shows the number of accepted flips during $10^5$ MC cycles plotted for temperature. As expected, few flips are accepted below the critical temperature in the ferromagnetic phase, on steady-state.That is why we see a constant line on $T=1$ For higher temperatures, the amount of accepted states are high, higher than $50\%$ due to the increase in disorder.\\\


As we can see from the probability distribution of figure 5 for the system with 20 x 20 lattice, the system with temperature T = 1 is more stable and converges towards an energy state that is lower than the system with T = 2.4 For the system with temperature of T = 2.4 it is harder to pin point an exact convergence, but we clearly see that it is most likely to find the system in an energy state of between -1.25 and -1.20. This makes sense since we add energy to the system when we raise the temperature.\\\



The top left picture of figure 6, shows that the energy increases with increasing temperature. This makes sense , because the structure gets more disordered.

This is expected, as the spin configuration becomes more disordered. We also see that the energy increases faster around the critical temperature for bigger lattices. The heat capacity seems to be bigger for large lattices around the critical temperature.

We think that the top right graph on figure 6 for absolute magnetism  show some reasonable results. This graph shows  that the critical temperature gets closer to the analytical value as the lattice size increases. We also see that the phase transition become more visible, in the sense that one would expect the slope of the order parameter to approach infinity at the critical temperature. The paramagnetic to ferromagnetic phase transition in a 2D Ising model on a square lattice is continuous at the critical temperature, which is in correspondence with the theory. As for susceptibility, it peaks around the critical temperatures, as expected.


Due to the overwelming calculation time , only 10 000 cycles where used for this project. The results could have been better with more cycles. This can clearly be seen from figure 2 and 3 since the numerical values get smaller and smaller after 10 000 cycles. 



\begin{figure}[H]
	\centering
	\includegraphics[width = \textwidth]{limit.png}
	\centering
	\caption{Linear Regression used to find the critical temperature as $L \to \infty$ can be interpreted as the constant term, which is $T_C(L\to\infty) \approx 2.287 J/k_B$. }
	\label{fig: integration limits Gauss-Legendre}
\end{figure}

Figure 8 was constructed by taking the maximum values we obtained with the different lattice sizes and their corresponding temperature. To obtain the maximal values, a filtering method known as as rolling mean was used. Then, a linear regression model was used one the plotted values to find the critical temperature. But again, Since we only used 10 000 MC cycles for this, we can not say with certainty that $T_C(L\to\infty) \approx 2.287 J/k_B$ is indeed the critical temperature. Because of the error with 10 000 cycles provides us, the critical temperature is most likely bellow the one we got. \\ \\ 

Besides parallellization, we also used compiler flags to further optimize our code. What we noticed when parallelizing was that we got on average 43 per cent better performance when using the -O2 compiler flag when it comes to running time. Figure 9 Shows a plot for how the running time compares with and without the -O2 compiler flag. (Again, because of bad performance on our machines, we did not go further than a lattice size of 50)...



		\begin{table}[h]
			\centering
			\caption{The average increase of speed shown by the table below.}
			\label{table: Error results Gauss-Quarature}
			\begin{tabular}{|c|c|c|c|}
\hline
\textbf{L} 	&	\textbf{no flag}	&	\textbf{-O2flag} & \textbf{Speed increase} \\\hline 
20  &  3.92506  &  2.00267   &       0.5102 \\\hline 
30   &  7.62552  &  3.66395    &      0.4804 \\\hline 
40   &  12.7915  &  5.4596   &        0.4268 \\\hline 
50   &  19.8124 &  7.95295   &        0.4014 \\\hline
			\end{tabular}
		\end{table}


\begin{figure}[H]
	\centering
	\includegraphics[width = \textwidth]{runtimeComparison.png}
	\centering
	\caption{ Plot demonstrating the comparison of the runtime with no flag and with O2 flag  }
	\label{fig: integration limits Gauss-Legendre}
\end{figure}


\section{Conclusion}

In this project we have studied the phase transition of a ferromagnetic material in a  two dimensional square lattice. The Curie temperature was found using Monte Carlo simulations with the Metropolis algorithm. Using the calculated values of 




\begin{equation}
T_C(L\to\infty) \approx 2.287J/k_B,
\end{equation}

By calculating the heat capacity for different lattice sizes, we could extract the critical temperature by locating the maximum of the heat capacity with the help of a filtering method and linear regression.. 



\begin{thebibliography}{9}

\bibitem{Ising} Ising, E., Beitrag zur Theorie des Ferromagnetismus,
	\emph{Z. Phys.,} 31, pp. 253-258 (1925).
\bibitem{hjorten} M. Hjorth-Jensen, Computational Physics Lecture Notes Fall 2019 (2019).

    \bibitem{onsager} L. Onsager, Phys. Rev. \textbf{65}, 117 (1944).


\end{thebibliography}


\end{document}

